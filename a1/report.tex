\documentclass[12pt,a4paper]{report}
\usepackage{amssymb}
\usepackage{amsthm}
\usepackage{amsmath}


\begin{document}
	
\section{Task 1}

The domain $[0, 1] \times [0, 1]$ is divided into elements by splitting each cell of a uniform Cartesian grid from the top left to the bottom right corner to make triangles.

\begin{figure}
	
\end{figure}

The reference element $\tilde{K}$ is a right triangle with the right corner in $(0, 0)$ and the other corners in $(0, 1)$ and $(1, 0)$ respectively. All affine functions over the element can be constructed from linear combinations of the three basis functions
\begin{align}
    \tilde{\phi}_1(x, y) = 1 - x - y \\
    \tilde{\phi}_2(x, y) = x \\
    \tilde{\phi}_3(x, y) = y
\end{align}
where each of the basis functions is nonzero in only one of the element`s corners.

On the reference element we have
\begin{align}
    \tilde{a_{ij}} = \int_{\tilde{K}} \nabla \tilde{\phi}_i \cdot \nabla \tilde{\phi}_j dx \\
    \tilde{m_{ij}} = \int_{\tilde{K}} \tilde{\phi}_i \tilde{\phi}_j dx \\
\end{align}
On the basis element, t

By

Continuous piece-wise affine functions on the whole domain are spanned by the 


	
\end{document}